\documentclass[a4paper, 12 pt]{article}

%Indice per la scaletta della tesi
\begin{document}
	\begin{center}
		\section* {Progetto di Ingegneria del Software Avanzata}
		\subsubsection*{\textit{Bertelli Alessandro}}
	\end{center}
	\break

	\section* {Contents}
	\begin{enumerate}
		\item Introduzione
		\item Descrizione del database
		\begin{itemize}
			\item Analisi dei requisiti con descrizione del mini-mondo
			\item Progettazione dello schema ER/EER
			\item Schema Relazionale
		\end{itemize}
		\item Requisiti di sistema
		\begin{itemize}
			\item Controllo di versione: \textit{Git/GitKraken}
			\item Gestione delle dipendenze: \textit{conda}
			\item Suite di test automatizzati: \textit{pytest-flask}
			\item Deployment: \textit{Docker}
		\end{itemize}
		\item Statechart
	\end{enumerate}
	\break
%-------------------------------------------------------------

	\section* {Introduzione}
	%Spiego brevemente cosa fa l'app
	
	\break
	
	\section* {Descrizione del database}
		\subsection*{Analisi dei requisiti con descrizione del mini-mondo}
		La base di dati PIZZERIA tiene traccia di tutte le prenotazioni di una pizzeria per asporto,
		dei menu delle pizze e delle bevande, dei clienti e dei coupon (buoni sconto) messi a disposizione
		dalla pizzeria.
		
		Ogni prenotazione ha un codice univoco, un cliente associato, un orario di riferimento e la lista dei
		codici dei prodotti (pizze e bevande) desiderati.
		I prodotti sono di due tipi: pizze o bevande.
		
		Ogni pizza ha un codice univoco, un nome, (la lista di ingredienti) e un prezzo.
		Ogni bevanda ha un codice univoco, un nome e un prezzo.
		Ogni cliente è identificato da un ID cliente univoco che sarà associato al nome e, in modo facoltativo, al numero di telefono del cliente.
		Inoltre la pizzeria in particolari occasioni rilascia dei coupon a qualche cliente permettendogli
		di avere diversi tipi di sconto.
		
		Ogni coupon ha un codice univoco identificativo e un valore in euro pari al valore dello sconto applicabile
		(chiaramente dopo essere stato utilizzato viene eliminato dal database, i coupon non sono cumulabili).
		
		\subsection*{Progettazione dello schema ER/EER}
		%Immagine schema ER completo (prima non normalizzato e poi quello di DBeaver)
		\subsection*{Schema Relazionale}
		%Schema relazionale non normalizzato e normalizzato
	\break
	
	\section* {Requisiti di sistema}
		\subsection*{Controllo di versione: \textit{Git/GitKraken}}
			\subsubsection*{Organizzazione repository}
			%Ci piazzo anche il link alla repository e foto GUI Kraken della repository
		\subsection*{Gestione delle dipendenze: \textit{conda}}
		%screen conda list proj_env
		\subsection*{Suite di test automatizzati: \textit{pytest-flask}}
		%screen terminale pytest flask e spiegazione dei test
		\subsection*{Deployment: \textit{Docker}}
	\break
	
	\section*{Statechart}
	\break
	
	%Ricordarsi di modificare il README
	
\end{document}
